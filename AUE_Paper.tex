\documentclass[sigchi]{acmart}

%%
%% \BibTeX command to typeset BibTeX logo in the docs
\AtBeginDocument{%
	\providecommand\BibTeX{{%
			\normalfont B\kern-0.5em{\scshape i\kern-0.25em b}\kern-0.8em\TeX}}}

%% Rights management information.  This information is sent to you
%% when you complete the rights form.  These commands have SAMPLE
%% values in them; it is your responsibility as an author to replace
%% the commands and values with those provided to you when you
%% complete the rights form.
\setcopyright{acmcopyright}
\copyrightyear{2018}
\acmYear{2018}
\acmDOI{10.1145/1122445.1122456}

%% These commands are for a PROCEEDINGS abstract or paper.
\acmConference[Woodstock '18]{Woodstock '18: ACM Symposium on Neural
	Gaze Detection}{June 03--05, 2018}{Woodstock, NY}
\acmBooktitle{Woodstock '18: ACM Symposium on Neural Gaze Detection,
	June 03--05, 2018, Woodstock, NY}
\acmPrice{15.00}
\acmISBN{978-1-4503-9999-9/18/06}


%%
%% Submission ID.
%% Use this when submitting an article to a sponsored event. You'll
%% receive a unique submission ID from the organizers
%% of the event, and this ID should be used as the parameter to this command.
%%\acmSubmissionID{123-A56-BU3}

%%
%% The majority of ACM publications use numbered citations and
%% references.  The command \citestyle{authoryear} switches to the
%% "author year" style.
%%
%% If you are preparing content for an event
%% sponsored by ACM SIGGRAPH, you must use the "author year" style of
%% citations and references.
%% Uncommenting
%% the next command will enable that style.
%%\citestyle{acmauthoryear}

%%
%% end of the preamble, start of the body of the document source.
\begin{document}
	
	%%
	%% The "title" command has an optional parameter,
	%% allowing the author to define a "short title" to be used in page headers.
	\title{Different interaction modalities for smart home}
	
	%%
	%% The "author" command and its associated commands are used to define
	%% the authors and their affiliations.
	%% Of note is the shared affiliation of the first two authors, and the
	%% "authornote" and "authornotemark" commands
	%% used to denote shared contribution to the research.
	\author{Fabian Hoffmann}
	\affiliation{%
		\institution{University of Regensburg}
		\city{Regensburg}
		\state{Bavaria}
		\country{Germany}
	}
	\email{fabian.hoffmann@stud.uni-regensburg.de}
	
	\author{Miriam Ida Tyroller}
	\affiliation{%
		\institution{University of Regensburg}
		\city{Regensburg}
		\state{Bavaria}
		\country{Germany}
	}
	\email{miriam-ida.tyroller@stud.uni-regensburg.de}
	
	\author{Felix Wende}
	\affiliation{%
		\institution{University of Regensburg}
		\city{Regensburg}
		\state{Bavaria}
		\country{Germany}
	}
	\email{felix.wende@stud.uni-regensburg.de}
	
	%%
	%% By default, the full list of authors will be used in the page
	%% headers. Often, this list is too long, and will overlap
	%% other information printed in the page headers. This command allows
	%% the author to define a more concise list
	%% of authors' names for this purpose.
	\renewcommand{\shortauthors}{Trovato and Tobin, et al.}
	
	%%
	%% The abstract is a short summary of the work to be presented in the
	%% article.
	\begin{abstract}
		We present a study aimed to gain insight on users' perceptions and desires in the context of smart home interaction techniques. To achieve this, we conducted an elicitation study in which participants were asked to perform commands within a simulated smart home environment, facing three conditions: voice command, display interaction and mid-air gestures. Facing tasks of different areas in smart home that require user assistance, the participants suggested fitting commands and rated them on the grounds of goodness, ease, enjoyment and social acceptance, as well as their general preference of each technique. The collected measures allow us to present insights that can be used as possible future guidelines for smart home interaction techniques and future research in voice command, display interactions and mid-air gestures.
	\end{abstract}
	
	%%
	%% The code below is generated by the tool at http://dl.acm.org/ccs.cfm.
	%% Please copy and paste the code instead of the example below.
	%%
	
	
	%%
	%% Keywords. The author(s) should pick words that accurately describe
	%% the work being presented. Separate the keywords with commas.
	\keywords{smart home, voice control, display control, mid-air gestures}
	
	
	%%
	%% This command processes the author and affiliation and title
	%% information and builds the first part of the formatted document.
	\maketitle
	
	\section{Introduction}
	While Smart Homes are widely known, they are not widely used. Possible reasons might be high costs, lack of understanding, worries about privacy, the lack of additional value, premature technologies and complicated installation \cite{.}. Solving these problems to further spread the use of smart homes can be done with two different approaches - either remove all the existing obstacles or develop smart home systems so desirable for its users they are not bothered by these obstacles any more, as developed by Hagensby Jensen et al. \cite{Jensen.2018}. Instead of using interaction modalities for smart homes, that are as simple and efficient as possible, the study's intend is to catch potential users attention to own a smart home, by featuring desirable and enjoyable traits.
	
	Our goal is to gain a set of voice commands, display interactions and mid-air gestures, to find out what users want and prefer to do, to solve different smart home tasks.
	
	\section{Approach and Methodology}
	We followed a similar approach as Dingler et al. \cite{Dingler.2018} by showing and explaining different smart home tasks to the participants and subsequently asking them to propose a voice command, a display interaction and a mid-air gesture, to fulfil the specific tasks in their preferred way. All eleven tasks are listed in section 'Tasks'. A within-subject design was chosen, so every participant gave suggestions for every modality and task. We used a latin-square on the order of the interaction modalities to reduce sequence effects \cite{.2017} and fatigue. The tasks were shown to the participants in  random order. We took video recordings of all sessions. We also collected feedback from participants through questionnaires, on preferences of interaction modalities for a specific task and on goodness, ease, enjoyment and social acceptance of their suggestions.
	
	\subsection{Interaction Modalities}
	We compared three different types of interaction modalities. The already commonly used voice and display control, as well as the existing but lacking development technique of mid-air gestures. Therefore, we were able to collect insights on the existing techniques and additionally gain a new set of mid-air gestures.
	
	\subsection{Tasks}
	The smart home market can be divided into six different categories \cite{.c}. Those are \textit{home entertainment, smart household appliances, energy management, networking and control, comfort and light} and \textit{building security}. We excluded the category \textit{networking and control} for developing the tasks, because it does not include devices that can be controlled, but is rather the infrastructure of a smart home and would be responsible for the detection of performed commands. For all other categories we selected two common tasks \cite{.d} each, except for \textit{building security} three because of its bigger market share. All categories with their assigned tasks are listed in table \ref{tab:tasks}.
	\begin{table}[t]
		\caption{Categories with their assigned tasks}
		\label{tab:tasks}
		\begin{small}
			\begin{tabular}{p{0.35\columnwidth} p{0.6\columnwidth}} \toprule
				\textbf{Category}			& \textbf{Task} \\ \midrule
				Home Entertainment         	& \begin{itemize}
					\item[1.] Increase the volume of the music.
					\item[2.] Turn on the next TV channel.  
				\end{itemize} \\ \midrule
				Smart household appliances 	& \begin{itemize}
					\item[3.] Start multi-colored wash at 60 degree.
					\item[4.] Turn off the oven.   
				\end{itemize} \\ \midrule
				Energy Management          	& \begin{itemize}
					\item[5.] Increase the room temperature.   
					\item[6.] Open the shutters.    
				\end{itemize} \\ \midrule
				Comfort and light          	& \begin{itemize}
					\item[7.] Turn on the light.      
					\item[8.] Dim the light.          
				\end{itemize} \\ \midrule
				Building security          	& \begin{itemize}
					\item[9.] Close the window.     
					\item[10.] Lock the front door. 
					\item[11.]  Turn on the security camera.
				\end{itemize} \\ \bottomrule
				
			\end{tabular}
		\end{small}
	\end{table}
	
	%\begin{table*}[]
	%\caption{Categories with their assigned tasks}
	%\label{tab:tasks}
	%\begin{tabular}{|l|p{0.4\textwidth}|}
	%\hline
	%\textbf{Category}          & \textbf{Tasks} \\ \hline
	%Home Entertainment         & \begin{itemize}
	%							 \item Increase the volume of the music.
	%							 \item Turn on the next TV channel.  
	%							 \end{itemize} \\ \hline
	%Smart household appliances & \begin{itemize}
	%							 \item Start multi-colored wash at 60 degree.
	%							 \item Turn off the oven.   
	%							 \end{itemize} \\ \hline
	%Energy Management          & \begin{itemize}
	%							 \item Increase the room temperature.   
	%							 \item Open the shutters.    
	%							 \end{itemize} \\ \hline
	%Comfort and light          & \begin{itemize}
	%							 \item Turn on the light.      
	%							 \item Dim the light.          
	%							 \end{itemize} \\ \hline
	%Building security          & \begin{itemize}
	%							 \item Close the window.     
	%							 \item Lock the front door. 
	%							 \item  Turn on the security camera.
	%							 \end{itemize} \\ \hline
	%\end{tabular}
	%\end{table*}
	
	\subsection{Participants}
	A total of 13 participants (7 female) took part in the study with an average age of $33.5$ (SD = $15.1$). We recruited the participants through social networks and personal contacts. The participants were mostly students from different departments of the University of Regensburg and OTH Regensburg. All of them at least heard of smart homes before and are familiar with interaction through displays. According to the pre-questionnaire, ten participants are familiar with both voice control and display interaction to control other devices, but only one performed mid-air gestures for interaction yet. Seven participants own smart home devices like Google Home, Amazon Alexa, smart TVs or lamps and use them frequently. None of the participants owns a fully integrated smart home system.
	
	\subsection{Apparatus}
	The study was carried out in a quiet room. The different tasks were illustrated through pictures, which showed the state before and after issuing the command. Mid-air gestures, voice commands and comments of the participants were recorded by a mounted camera. Display interaction was documented through a sketch on paper. None of the interaction modalities were actually implemented.
	
	\subsection{Procedure}
	Before starting the session, the participants were asked to fill out a consent form and a demographic questionnaire. Then they had to fill out a questionnaire in terms of their previous knowledge and usage of smart home devices and the three interaction modalities. After that the tasks were presented to the participants in a random order. At first all tasks had to be fulfilled with a single interaction modality, then with the second and after that with the remaining modality. Additionally to the illustration through pictures, the tasks were explained verbally. The participants were allowed to talk, move and interact with a display in any way they wanted and were encouraged to explain their choices in a thinking-aloud approach. After each task the participants rated their specific suggestion on goodness, ease, enjoyment and social acceptance on four 7-point Likert scales. When all tasks were finished with each interaction modality the participants rated the three interaction modalities for each on 7-point Likert scales, on how good each modality is to perform the specific task. They were asked to do this independently of their own suggestions. At the end a semi-structured interview was conducted to explore the motivation of the participants for each choice and allow them to rate the different interaction modalities under the aspects of efficiency, simplicity, naturality, desirability and enjoyment. this is based on a similar approach in the elicitation study on foot gestures by Felberbaum et al. \cite{Felberbaum.2018}. The study took about an hour, for which the participants were compensated with sweets.
	
	\section{Results}
	With nine participants and eleven tasks, we collected for each of the three interaction modalities $143$ suggestions and in total $13*11*3=429$. Our results include the video recording, taxonomies for each interaction modality, user-defined sets of voice commands, display interactions and gestures, subjective ratings of the sets, qualitative observations and an assessment on the modalities for each task. 
	
	\subsection{Classification of Voice Commands}
	\subsubsection{Taxonomy of Voice Commands}
	The participants suggested $43$ unique voice commands. The authors manually classified each voice command along five dimensions: \textit{nature}, \textit{form}, \textit{flow}, \textit{context} and \textit{complexity}. Within each dimension are multiple categories, shown in Table \ref{tab:taxVoice}. We adopted the dimensions from Wobbrock \textit{et al.} \citep{Wobbrock.2009} and Ruiz \textit{et al.} \citep{Ruiz.2011} and adapted them to voice commands. 
	
	The \textit{nature} dimension compromises \textit{action} voice commands which state the action to perform. An example of this type of voice command is saying "increase temperature". \textit{State} voice commands describe the desired condition of a device. For example, a \textit{state} voice command is "cameras on" to start camera surveillance.
	
	The \textit{form} dimension describes how much words are used in the voice command and if they have the structure of a full sentence. A \textit{single word} command can be "next" to get to the next TV channel. \textit{Two words} voice commands mostly consist out of the mentioning of the device to be controlled and an action or state. \textit{More words} commands are similar to \textit{two words} but use additional filler words. Finally, voice commands that are correct sentences were classified with the category \textit{sentence}.
	
	The \textit{flow} dimension categorizes the voice commands, if response of a device occurs after or while the user acts. A voice command is\textit{discrete}, when device perform the command after the participant stopped talking. A \textit{continuous} voice command would be starting an action with a command and stop the ongoing action with another command.
	
	The \textit{context} dimension describes, if the voice command requires a specific context or can be performed independently. For example saying "turn off" to turn off the oven is \textit{in-context}, whereas "oven off" is considered \textit{no-context}.
	
	The \textit{complexity} dimension describes if the voice command consists out of a single or a composition of more voice commands. A \textit{compound} voice command can be decomposed into \textit{simple} voice commands.
	\begin{table}[t]
		\begin{center}
			\caption{Taxonomy of voice commands for smart home tasks}
			\label{tab:taxVoice}
			\begin{small}
				\begin{tabular}{p{0.2\columnwidth} p{0.2\columnwidth} p{0.5\columnwidth}} \toprule
					\multicolumn{3}{c}{\textit{Taxonomy of voice commands}} \\ \midrule
					\textbf{Nature}		& Action		& Voice command states the action to perform \\
					& State			& Voice command describes the desired condition \\ \midrule
					\textbf{Form} 		& Single word	& Voice command consists out of a single word \\
					& Two words		& Voice command consists out of two words \\
					& More words	& Voice command consists out of more words without sentence structure \\
					& Sentence		& Voice command uses sentence structure \\ \midrule
					\textbf{Flow}		& Discrete		& Response occurs \textit{after} the user acts \\ 
					& Continuous 	& Response occurs \textit{while} the user acts  \\ \midrule
					\textbf{Context}	& In-context	& Voice command requires specific context \\
					& No-context	& Voice command does not require specific context \\ \midrule
					\textbf{Complexity}	& Simple		& Voice command consists of a single voice command \\
					& Compound		& Voice command can be decomposed into simple voice commands  \\ \bottomrule
				\end{tabular}
			\end{small}
		\end{center}
	\end{table}
	\subsection{User-defined voice command set}
We collected a total of $143$ voice commands, which we used to create a user-defined voice command set for our specified tasks. For each task, we grouped identical voice commands together and the group with largest size was chosen to be the representative voice command for this corresponding task. To evaluate the degree of consensus among the participants, we computed the \textit{agreement score} $A_t$ (Equation \ref{agreement score}), as proposed by Vatavu and Wobbrock \cite{Vatavu.2015}, for each task.
\begin{equation}
\label{agreement score}
	A_t = \frac{|P_t|}{|P_t|-1} \sum_{P_i \subseteq P_t} \left(\frac{|P_i|}{|P_t|}\right)^2  - \frac{1}{|P_t|-1}
\end{equation}
In equation \ref{agreement score}, $t$ is a task in the set of all tasks $T$, $P_t$ is the set of suggested voice commands for $t$ and $P_i$ is a subset of identical voice commands from $P_t$.
As an example of calculation of an agreement score, consider the task \textit{increase the volume of the music}. The task has four groups of identical voice commands with a size of $7$, $4$, $1$ and $1$. Therefore the agreement score for \textit{increase the volume of the music} is:
\begin{equation}
	A = \frac{13}{12} \left(\left(\frac{7}{13}\right)^2 + \left(\frac{4}{13}\right)^2 + \left(\frac{1}{13}\right)^2 + \left(\frac{1}{13}\right)^2 \right)- \frac{1}{12} = 0.346
\end{equation}
	\subsection{Classification of Display Interaction}
	\subsection{Classification of Mid-Air Gestures}
	blabla
	
	
	
	
	
	%%
	%% The next two lines define the bibliography style to be used, and
	%% the bibliography file.
	\bibliographystyle{ACM-Reference-Format}
	\bibliography{AUE_Paper}
	
	%%
	%% If your work has an appendix, this is the place to put it.
	\appendix
	
	\section{Research Methods}
	
	\subsection{Part One}
	
	Lorem ipsum dolor sit amet, consectetur adipiscing elit. Morbi
	malesuada, quam in pulvinar varius, metus nunc fermentum urna, id
	sollicitudin purus odio sit amet enim. Aliquam ullamcorper eu ipsum
	vel mollis. Curabitur quis dictum nisl. Phasellus vel semper risus, et
	lacinia dolor. Integer ultricies commodo sem nec semper.
	
	\subsection{Part Two}
	
	Etiam commodo feugiat nisl pulvinar pellentesque. Etiam auctor sodales
	ligula, non varius nibh pulvinar semper. Suspendisse nec lectus non
	ipsum convallis congue hendrerit vitae sapien. Donec at laoreet
	eros. Vivamus non purus placerat, scelerisque diam eu, cursus
	ante. Etiam aliquam tortor auctor efficitur mattis.
	
	\section{Online Resources}
	
	Nam id fermentum dui. Suspendisse sagittis tortor a nulla mollis, in
	pulvinar ex pretium. Sed interdum orci quis metus euismod, et sagittis
	enim maximus. Vestibulum gravida massa ut felis suscipit
	congue. Quisque mattis elit a risus ultrices commodo venenatis eget
	dui. Etiam sagittis eleifend elementum.
	
	Nam interdum magna at lectus dignissim, ac dignissim lorem
	rhoncus. Maecenas eu arcu ac neque placerat aliquam. Nunc pulvinar
	massa et mattis lacinia.
	
\end{document}
\endinput
%%
%% End of file `sample-sigchi.tex'.